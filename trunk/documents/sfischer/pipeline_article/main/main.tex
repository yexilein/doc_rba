\documentclass{bioinfo}
\copyrightyear{2015} \pubyear{2015}

\access{Advance Access Publication Date: Day Month Year}
\appnotes{Manuscript Category}

\begin{document}
\firstpage{1}

\subtitle{Subject Section}

\title[short Title]{\texttt{rba} package for Resource Balance Analysis}
\author[Sample \textit{et~al}.]{
  Corresponding Author\,$^{\text{\sfb~1,}*}$,
  Co-Author\,$^{\text{\sfb~2}}$ and
  Co-Author\,$^{\text{\sfb~2,}*}$
}
\address{$^{\text{\sf 1}}$Department, Institution, City, Post Code, Country and \\
$^{\text{\sf 2}}$Department, Institution, City, Post Code,
Country.}

\corresp{$^\ast$To whom correspondence should be addressed.}

\history{Received on XXXXX; revised on XXXXX; accepted on XXXXX}

\editor{Associate Editor: XXXXXXX}

\abstract{\textbf{Motivation:} Text Text Text Text Text Text Text Text Text Text Text Text Text
Text Text Text Text Text Text Text Text Text Text Text Text Text Text Text Text Text Text Text
Text Text Text Text Text Text Text Text Text Text Text Text Text Text Text Text Text Text Text
Text Text Text Text Text Text
Text Text Text Text Text.\\
\textbf{Results:} Text  Text Text Text Text Text Text Text Text Text  Text Text Text Text Text
Text Text Text Text Text Text Text Text Text Text Text Text Text  Text Text Text Text Text Text\\
\textbf{Availability:} Text  Text Text Text Text Text Text Text Text Text  Text Text Text Text
Text Text Text Text Text Text Text Text Text Text Text Text Text Text  Text\\
\textbf{Contact:} \href{name@bio.com}{name@bio.com}\\
\textbf{Supplementary information:} Supplementary data are available at \textit{Bioinformatics}
online.}

\maketitle

\section{Introduction}

Metabolism is traditionally explored with Flux Balance Analysis (FBA).
FBA blah blah blah.

Resource Balance Analysis (RBA) was introduced by \citet{goelzer}.
RBA is build around three principles.
\begin{itemize}
  \item Stoichiometry constraints (similar to FBA) for mass conservation.
  \item Machinery costs (enzymes, ribosomes) that associate costs
  to metabolic pathways and other cell processes.
  \item Density constraints preventing infinite growth.
\end{itemize}
In RBA, cells select pathways based on their
metabolic costs in order to achieve maximal growth rates.
A typical illustration of this phenomenon is catabolic repression~\citep{mimi}.

\begin{methods}
\section{Methods}

In this paper, we present the \texttt{rba} Python packag.
\texttt{rba} was designed to perform three tasks.
\begin{itemize}
  \item Build a working RBA model from an SBML file and Uniprot data.
  A RBA model can be exported to a set XML files (see Supplementary Material).
  \item Solve a RBA model.
  \item Visualize solutions.
\end{itemize}

\end{methods}

\begin{figure}[!tpb]%figure1
\centerline{\includegraphics{OUP_First_SBk_Bot_8401.eps}}
\caption{Caption, caption.}\label{fig:01}
\end{figure}

\section{Illustration}

Illustration of prerba and nice output for subtilis/coli.

\section{Conclusion}

\section*{Acknowledgements}

\section*{Funding}

This work has been supported by the Text Text  Text Text.\vspace*{-12pt}

\bibliographystyle{natbib}
%\bibliographystyle{achemnat}
%\bibliographystyle{plainnat}
%\bibliographystyle{abbrv}
%\bibliographystyle{bioinformatics}
%
%\bibliographystyle{plain}
%
\bibliography{main}


\end{document}
