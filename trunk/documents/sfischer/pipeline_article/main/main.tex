\documentclass{bioinfo}
\copyrightyear{2015} \pubyear{2015}

\access{Advance Access Publication Date: Day Month Year}
\appnotes{Manuscript Category}

\begin{document}
\firstpage{1}

\subtitle{Subject Section}

\title[RBAb pipeline]{RBAb: a pipeline to build genome-scale cellular models}
\author[Fischer \textit{et~al}.]{
  Stephan Fischer\,$^{\text{\sfb 1}}$,
  Ana Bulovic\,$^{\text{\sfb 2}}$,
  Marc Dinh\,$^{\text{\sfb 1}}$,
  Vincent Fromion\,$^{\text{\sfb 1}}$
  and Anne Goelzer\,$^{\text{\sfb 1},*}$
}
\address{
  $^{\text{\sf 1}}$ UR1404 MaIAGE, INRA, University of Paris-Saclay,
    78350 Jouy-en-Josas, France, and
  $^{\text{\sf 2}}$ Theoretical Biophysics, Humboldt-Universität zu Berlin,
    Berlin, Germany
}

\corresp{$^\ast$To whom correspondence should be addressed.}

\history{Received on XXXXX; revised on XXXXX; accepted on XXXXX}

\editor{Associate Editor: XXXXXXX}

\abstract{
\textbf{Motivation:}
Yes\\
\textbf{Results:}
No\\
\textbf{Availability:}
The software is freely available and can be downloaded from\\
\textbf{Contact:} \href{anne.goelzer@inra.fr}{anne.goelzer@inra.fr}\\
\textbf{Supplementary information:}
Supplementary data are available at \textit{Bioinformatics} online.
}

\maketitle

\section{Introduction}
Parcimonious resource allocation emerged as a strong design principle
for microorganisms \citep{goelzer_cell_2009,scott_interdependence_2010}.
Based on this principe, new types of genome-scale cellular
constraint-based models have been developed
\citep{goelzer_cell_2011,obrien_genomescale_2013}.
In particular, we developed the resource balance analysis (RBA) method
that has been recently validated for the bacterium
\textit{Bacillus subtilis} \citep{goelzer_quantitative_2015}.

RBA models integrate 3 set of constraints:
mass balance in the metabolic network such as
in genome-scale metabolic models \citep{varma_stoichiometric_1994},
capacity constraints on cellular processes and
density constraints on the cytosol and membrane occupancy
(reviewed in \citet{goelzer_resource_2017}).
RBA models necessitate various sources of information on cellular economics
(composition of molecular entities, processing costs and composition
of cellular processes, etc.)
and quantitative parameters related to cell physiology
or to the efficiency of molecular machineries \citep{goelzer_resource_2017}.

In this paper, we present a pipeline that builds
a valid RBA model from an SBML file.
Information about molecular composition is retrieved in
a semi-automated process.
The pipeline generates a model containing basic cell processes.
The model is written in XML files to enable further enrichments.

\section{Source of information}
Resource allocation models require various sources of information.

\section{Pipeline structure}

We present the \texttt{rba} Python package.
\texttt{rba} was designed to perform three tasks.
\begin{itemize}
  \item Build a working RBA model from an SBML file and Uniprot data.
  A RBA model can be exported to a set XML files (see Supplementary Material).
  \item Solve a RBA model.
  \item Visualize solutions.
\end{itemize}


\begin{figure}[!tpb]%figure1
\centerline{\includegraphics{OUP_First_SBk_Bot_8401.eps}}
\caption{Caption, caption.}\label{fig:01}
\end{figure}

\section{Illustration}

Illustration of prerba and nice output for subtilis/coli.

\section{Conclusion}

\section*{Acknowledgements}

\section*{Funding}

This work has been funded by the French Lidex-IMSV of the University Paris-Saclay.\\
\textit{Conflict of Interest}: none declared.
\vspace*{-12pt}

\bibliographystyle{natbib}
%\bibliographystyle{achemnat}
%\bibliographystyle{plainnat}
%\bibliographystyle{abbrv}
%\bibliographystyle{bioinformatics}
%
%\bibliographystyle{plain}
%
\bibliography{main}


\end{document}
