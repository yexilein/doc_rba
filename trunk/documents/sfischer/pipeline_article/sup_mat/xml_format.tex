% ============================================
%  Article Class (This is a LaTeX2e document)
% ============================================
\documentclass[12pt]{scrartcl}
\usepackage[english]{babel}
\usepackage{enumitem}
\usepackage[round]{natbib}
\usepackage{color}

\newcommand\reft[3][]{#2~\ref{#3}#1}
\newcommand\refp[3][]{(#2~\ref{#3}#1)}
\newcommand\refsect[1]{\reft{Section}{#1}}
\newcommand\refsecp[1]{\refp{Sec.}{#1}}
\newcommand\reftabt[1]{\reft{Table}{#1}}
\newcommand\reftabp[1]{\refp{Tab.}{#1}}

% ============
%  Algorithms
% ============
\usepackage{algorithm2e}
\SetKwProg{Fn}{Function}{}{}
\newcommand\refalgt[1]{\reft{Algorithm}{#1}}
\newcommand\refalgp[1]{\refp{Alg.}{#1}}

% ======
%  Math
% ======
\usepackage{amsmath}
\usepackage{amsthm}
\newtheorem{thm}{Theorem}[section]
\newtheorem{cor}[thm]{Corollary}
\newtheorem{lem}[thm]{Lemma}
\newtheorem{prop}[thm]{Proposition}
\newtheorem{property}[thm]{Property}
\theoremstyle{definition}
\newtheorem{defn}[thm]{Definition}
\newtheorem{assum}[thm]{Assumption}
\theoremstyle{remark}
\newtheorem{rem}[thm]{Remark}
\numberwithin{equation}{section}
\usepackage{amssymb}
\newcommand{\prob}[1]{\mathbb{P}\left(#1\right)}

% ============================
%  Figures and relative paths
% ============================
\usepackage{graphicx}
\graphicspath{{figures/}}
\usepackage{import}
\makeatletter
  \def\relativepath{\import@path}
\makeatother
\newcommand\reffigt[2][]{\reft[#1]{Figure}{#2}}
\newcommand\reffigp[2][]{\refp[#1]{Fig.}{#2}}

% ==========
%  Document
% ==========
\begin{document}

\title{XML format for RBA models}
\author{S. Fischer, V. Fromion, A. Goelzer}
\date{\today}

\maketitle

\newpage

\tableofcontents

\newpage

\section{Introduction}

In this document we present the XML structures used to define a RBA model.
A complete RBA model is composed of the following files:
\begin{itemize}
  \item metabolism.xml
  (definition of compartments, metabolic species and metabolic reactions).
  \item parameters.xml
  (definition of density constraints and user-defined functions).
  \item proteins.xml (definition of proteins).
  \item rnas.xml (definition of RNAs).
  \item dna.xml (definition of DNA).
  \item enzymes.xml
  (definition of enzymatic machineries catalyzing metabolic reactions).
  \item processes.xml
  (definition of cell processes necessary to growth and maintenance).
\end{itemize}

For every file, we present the nodes that composes the XML structure.
For every node, we show a class diagram that shows the node's attributes
and the children node that it may/must contain.
We provide a brief description about the relevance of the node
in the RBA model.

\section{\texttt{metabolism.xml}}

\end{document}
% ----------------------------------------------------------------
