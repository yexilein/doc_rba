
\section{Work in progress}

\subsection{Harmonize enzymes and process machineries}

Currently every reaction must be catalyzed by a different enzyme.
In comparison, process machineries, such as ribosomes, may catalyze multiple
production reactions.
We may use a similar system for enzymes where one enzyme may catalyze several
reactions.

In this case, the reaction to enzyme mapping becomes a little more complicated
and must take into account
\begin{itemize}
  \item Different k\_cat values for different reactions.
  \item Distinguish forward and backward reactions.
\end{itemize}
A side document explains how enzyme capacity constraints may be rewritten
in order to be closer to process capacity constraints.

Implementing the new mapping yields smaller matrices, as we need less
constraints overall.

\subsection{Generalize density constraints}

Instead of having one density constraint per compartment,
it would be nice to have constraints for linear combinations of compartments.
This would enable to define global density constraints, i.e,
the sum of weights in all compartments has to be lower or equal to some limit.

\subsection{Matrix conditioning and objective function}

Performance is extremely variable because our matrix is not well conditioned.
Sometimes, adding zero rows and columns drastically improves convergence
(not sure why, cplex eliminates these rows and columns during presolving, but
somehow converges in fewer iterations, maybe a bigger matrix increases
convergence tolerance?).
Finding a good matrix scaling and objective function may affect performance
significantly.
